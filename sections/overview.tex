\section{Observación General.}

El libro \textit{Non-Newtonian Calculus} de Grossman y Katz se asemeja más a un cuento de hadas que a un ensayo de matemáticas. Comienza lentamente, invirtiendo los primeros cuatro capítulos en introducir una serie de sistemas formales familiares a la par que anómalos para el estudiante de cálculo. Insiste en \textit{patrones} y \textit{repetición}, tanto es así que para el tercer capítulo uno empieza a cuestionarse si es posible obtener más rédito de la lectura. Sin embargo, hay un aroma a misticismo y magia entre sus páginas. Cierto rocío se desprende de la suma cuando se transforma en multiplicación y esta en exponenciación.

Cuando al fin el lector está al borde de darse cuenta de lo que está ocurriendo, el libro salta a un concepto que a priori nada tiene que ver: las aritméticas. Más que decepción, esto ofrece un merecido descanso de la serie aparentemente infinita de cálculos presentada. Cuando uno menos se lo espera, el gran giro de trama ocurre. El Capítulo 6 funde todas las entradas anteriores en una magnífica generalización de todos los principios vistos hasta ahora usando conceptos del Capítulo 5. El orgasmo parece una broma comparado con el placer matemático recibido. 

De aquí en adelante, el libro se puede leer en menos de una hora. Equipados con la todopoderosa generalización, uno empieza a explorar casos particulares del espacio infinito de cálculos. También se ofrecen usos prácticos, representaciones gráficas e ideas de cómo esto se puede aplicar para derivar todo tipo de matemáticas (números complejos, geometrías...).

Este libro es una exposición excelente sobre cómo las matemáticas son descubiertas, cómo se desarrollan y cómo deberían ser explicadas a otros. La matemática es la sofisticación de la fuerza bruta. Cuando se presenta un problema, soluciones toscas y a menudo sucias surgen, habitualmente por prueba y error. La mente humana es capaz de detectar fenómenos comunes en soluciones suficientemente similares, y a través de sus propias estructuras cognitivas y con la ayuda de representaciones visuales (sean gráficas o simbólicas), generalizar estas soluciones en una sola expresión o procedimiento. Esa noción general a menudo sirve como solución a otros problemas cuya similitud pasó inadvertida anteriormente. (Este algoritmo provée una respuesta a la irritante pregunta de si las matemáticas son descubiertas o inventadas).

Esta además resulta estar entre las mejores rutas para explicar matemáticas a aquéllos poco familiares con ellas. Partir de problemas reales, resolverlos por fuerza bruta, encontrar patrones y repetición, generalizar y aplicar la generalización a otros casos. Esto no solo recupera el componente humano de las matemáticas, sino que además las hace divertidas. Convierte el desarrollo de un teorema en algo parecido a ver una película de \textit{thriller}.

Esto (tanto el libro como la manera de enseñar) también ofrece algo de perspectiva sobre cómo la mayoría de las matemáticas han nacido y sido usadas. Al principio una pieza de matemáticas puede ser tan aburrida como cualquier cosa, pero dado el suficiente tiempo para jugar con ella --- axiomatizándola, derivando demostraciones, generalizando los conceptos... --- nuestro cerebro de primates puede llegar a encontrarla incluso entretenida, aunque solo sea como preliminar intelectual. Casi por pura coincidencia, esta inútil pero algo interesante masa de conceptos resulta ser increíblemente aplicable a otros campos de conocimiento. Por nombrar algunos ejemplos del libro, la suma harmónica puede encontrarse en la resistencia de resistores paralelos, los cálculos geométricos están por toda la economía, la media cuadrática se reduce a la RMS (\textit{root mean square}), etc.
