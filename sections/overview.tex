\section{Overview.}

\textit{Non-Newtonian Calculus} by Grossman and Katz struck me as a book more similar to a fairy-tale than a math essay. It starts slow, spending the first four chapters introducing a series of formal systems familiar and yet anomalous to the calculus student. It insists on \textit{patterns} and \textit{repetition}, so much so that by the third chapter one could think no more insight is to be obtained by continuing to read. However, there's a smell of mysticism and magic among those pages. A sprinkle addition leaves when morphing into multiplication and multiplication into exponentiation.

Just as the reader is on the brink of figuring out what's going on, the book jumps into the seemingly unrelated concept of arithmetics. More than disappointing, this offers a well-deserved break from the apparently infinite series of calculi presented. When one least expects it, the big plot-twist comes up. Chapter 6 merges all previous entries into a great generalization of every principle seen thus far using the concepts from Chapter 5. Orgasm is but a joke compared to the mathematical pleasure received.

From here on out, the book can be read in less than an hour. Equipped with the all-powerful generalization, one starts to explore particular cases from the infinite space of calculi. Real use-cases, graphical representations and how this can be applied to derive all kinds of math (complex numbers, geometries...) are also offered.

The book is an excellent exposition on how math is discovered, how it develops and how it should be explained to others. Mathematics is all about the sophistication of bruteforcing. When faced with a problem, rough, often dirty solutions come up, usually by trial and error. The human mind is able to detect common phenomena in solutions that are similar enough, and through it's own cognitive structures and the aid of visual representations (whether that be graphs or symbols), generalize those solutions into a single expression or procedure. That general notion often serves as the solution to other problems whose similarity went unnoticed previously. (This algorithm gives an answer to the annoying question of whether math is discovered or invented).

This also turns out to be among the best ways of explaining math to those not familiar with it. Start from real problems, bruteforce them, find patterns and repetition, generalize and apply to other cases. Not only this recovers the human component of math, but it also makes it outright fun! It makes developing theorems similar to watching a thriller movie.

It (both the book and the way of teaching) also gives some insight into how most mathematics' have been born and used. At first a piece of math might be as boring as anything, but when given enough time to toy around with it --- axiomatizing it, making proofs, generalizing the concepts... --- our monkish brain might as well find it entertaining, at least as intellectual foreplay. Almost by sheer coincidence, this useless but sort-of-interesting blob of concepts turns out to be massively applicable to other fields of knowledge! To name a few examples from the book, harmonic addition can be found on parallel resistors' resistance, the geometric calculi are all over economics, the quadratic average reduces into the root mean square, etc.
