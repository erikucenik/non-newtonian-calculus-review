\section{Afterthoughts.}

There's not much more to be said that won't be categorized as pretentious or rambling. We prefer to include Boscovich's quote which also appears at the beginning of the book.

\begin{quote}{Roger Joseph Boscovich}
	\enquote{\textit{We fashion our geometry on the properties of a straight line because that seems to us to be the simplest of all. But really all lines that are continuous and of a uniform nature are just as simple as one another. Another kind of mind which might form an equally clear mental perception of some property of any one of these curves, as we do of the congruence of a straight line, might believe these curves to be the simplest of all, and from that property of these curves build up the elements of a very different geometry, referring all other curves to that one, just as we compare them to a straight line. Indeed, these minds, if they noticed and formed an extremely clear perception of some property of, say, the parabola, would not seek, as our geometers do, to} rectify \textit{the parabola, they would endeavour, if one may coin the expression, to} parabolify \textit{the straight line}.}
\end{quote}

Finally, thanks to Sofia T. for letting me know and borrow this piece.
