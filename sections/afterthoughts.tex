\section{Comentarios Adicionales.}

No hay mucho más que pueda ser dicho sin ser categorizado de pretencioso o divagador. Preferimos incluir la cita de Roger Joseph Boscovich también presentada al principio del libro.

\begin{quote}{Roger Joseph Boscovich}
	\enquote{\textit{Construímos nuestra geometría sobre las propiedades de la línea recta porque esta nos parece la más simple de todas. Pero en realidad todas las líneas continuas con naturaleza uniforme son igual de simples. Otro tipo de inteligencia capaz de formar una percepción mental igual de clara sobre alguna propiedad de cualquiera de estas curvas, como nosotros lo hacemos con la congruencia de la línea recta, podría creer que sus curvas son las más simples de todas, y a partir de ellas construir los elementos de una geometría muy diferente, refiriendo todas las demás curvas a esa, al igual que nosotros las comparamos a la recta. Efectivamente, estas inteligencias, al notar una percepción extremadamente clara de alguna propiedad de, por ejemplo, la parábola, no buscarían, como lo hacen nuestros geómetras,} rectificar \textit{la parábola, sino que buscarían, si uno puede usar la expresión,} parabolificar \textit{a la recta}.}
\end{quote}

Por último, gracias a Sofia T. por prestarme y hacerme saber de esta pieza.
